\documentclass{article}
\usepackage[utf8]{inputenc}
\usepackage{amsmath,subfiles}
\usepackage{enumitem}
\usepackage{mathtools}
\usepackage{tikz-cd}
\usepackage{amssymb}
\usepackage[makeroom]{cancel}
\usepackage{graphicx}
\usepackage{pgfplots}
\usepackage{parskip}
\usepackage{hyperref}
\setlength{\parindent}{0pt}
\pgfplotsset{compat = newest}
\title{MD Simulation Notes}
\author{Armaan Ahmed}
\date{April 2023}

\begin{document}

\maketitle

A basic MD Simulation is based off of Newtonian physics:

\[ma=F\]

For a molecule $i$, need to sum over all the pairwise interactions. Then can numerically integrate or use:

\[\frac{x(t+dt)-x(t)}{dt} = v(t)\]

For solated system, there are too many particles. So, we use implicit solvent:

\[ma=F+F_{drag}+F_{noise}\]

The drag and noise encapsulate the effect of the solvent on the particle (ie drag force on the molecule). Essentially water is treated as just a continuum. Noise term accounts for the brownian motion, which is the ``kick'' given off by the water molecules. With implicit solvent, we need to use more parameteres, ie what is viscocity of water.

TwoChain\_np.chain specifies the number of bonds, number of molecules, locations, velocities

``Atoms'':

Index,chain,atom\_type,

``Velocities'':

``Bonds'':

Index,bond\_type,bonded\_atom\_1,bonded\_atom\_2.

Can also define bond angle, but optional.

In ``in\_np.chain'' file:

Units: scale
Boundary considerations
Atom\_style: specifies the connections

read\_data: topology of system

Bond\_style: what type of connection (normally has harmonic potential)

Bond coeff: parameters of the bond.

special\_bonds: get rid of pairwise interaction between bonded pairs.

Eg, lj 0.0 1.0 1.0

lj is a special parameter

pair\_style: cutoff distance for the interactions.

\end{document}
